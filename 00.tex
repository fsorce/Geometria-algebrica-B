\chapter*{Introduzione}


\section*{Scopo del corso}
Il filo conduttore che guida il corso \`e la seguente domanda:
\begin{center}
Quando una variet\`a complessa \`e algebrica?
\end{center}

Per \textbf{variet\`a complessa} si intende uno spazio che localmente si comporta come $\C^n$, mentre per \textbf{variet\`a algebrica} si intende il luogo di zeri di un sistema di equazioni polinomiali omogenee in\footnote{In questo corso $\Pj^n$ \`e sempre $\C\Pj^n=\Pj^n_\C(\C)$ se non diversamente specificato.} $\Pj^n$.

\medskip

Grazie al teorema di Chow, che vedremo durante il corso, possiamo trasformare la domanda in
\begin{center}
    Per quali variet\`a complesse $M$ esiste un embedding $M\inj \Pj^n$?
\end{center}

Vedremo che mappe olomorfe $f:M\to \Pj^n$ (pi\`u precisamente $f:M\ratto \Pj^n$) corrispondono a line bundle su $M$, quindi la domanda si riduce a capire per quali variet\`a complesse esistono line bundle che inducono un embedding.

\bigskip

Kodaira riesce a ricondurre questa domanda all'annullamento di alcuni gruppi di coomologia, il culmine sono i teoremi di Kodaira vanishing e Kodaira embedding, il secondo risponde in modo perlopi\`u soddisfacente alla domanda:
\begin{theorem}[Kodaira embedding]
Se $M$ \`e una variet\`a complessa compatta allora essa si immerge in $\Pj^n$ per qualche $n$ se e solo se esiste su $M$ una $(1,1)$-forma differenziale razionale, chiusa e positiva.
\end{theorem}
